\section{Affichage de texte}
\input{python/base/affichage_texte.tex}

\section{Commentaires}
\input{python/base/commentaires.tex}

\section{Variables}
\input{python/base/variables.tex}

\section{Le type \texttt{str}}
\input{python/base/string.tex}

\subsection{Exercices}
\begin{enumerate}
    \item \textbf{Exercice 1 :} Faites un programme qui affiche toutes les lettres du \texttt{String} "Python forever".
    \item \textbf{Exercice 2 :} Concaténer les \texttt{String} "We are", "the knights", "who say nih !" et affichez le résultat en majuscule ainsi que sa longueur.
\end{enumerate}

% \begin{table}[h!]
%     \centering
%     \begin{tabular}{|c|c|}
%         \hline
%          Opération & Description\\
%         \hline
%         len(nom\_string) & Calcule la longueur d'un String \\
%         \hline
%         nom\_string.lower() & Retourne le string en minuscule \\
%         \hline
%         nom\_string.upper() & Retourne le string en majuscule \\
%         \hline
%          str(nom\_variable) & Cast en String  \\
%         \hline
%         nom\_string1+nom\_string2  & Retourne les 2 String concaténés \\
%         \hline
%     \end{tabular}
%     \caption{Opérations String}
%     \label{operStr}
% \end{table}
