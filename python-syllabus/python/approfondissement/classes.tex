Avant d’avancer plus en profondeur, il est utile de définir quelques mots de vocabulaire.
\begin{itemize}
    \item Une classe est une définition d’un concept. Elle peut être agrémentée d’attributs et de
fonctions pour la caractériser.
    \item Un objet est une instance d’une classe, c'est-à-dire que son comportement est défini par la
classe.
\end{itemize}

Si l’on prend l’exemple d’un véhicule, son plan de fabrication et de fonctionnement correspond à la classe.
Une voiture concrète, qui possède ses propres caractéristiques, correspond à un objet de cette classe. Et on peut construire plusieurs voitures à partir d'un seul plan, et les personnaliser en changeant ses options (attributs!).

Une classe possède des attributs qui sont eux-mêmes des objets (de différents types), et des méthodes qui sont des fonctions qui s'appliquent sur les objets de cette classe.
Le mot-clef \texttt{self} permet de référencer l'objet sur lequel on est en train de travailler.
Il est implicitement passé en argument à toutes les méthodes.
Le constructeur est une méthode particulière de la classe : il permet de créer un nouvel objet.

\begin{python}[caption = Exemple de classe]
class Vehicule:

    """Attributs"""
    
    #Valeurs par defaut
    nombreRoues = 4
    nombrePortes = 5
    nombrePlaces = 5
    couleur = 'noir'
    kilometres = 0
    
    """Methodes"""
    
    #Constructeur, permet de creer un objet vehicule avec un certain prix
    def __init__(self, prix):
        self.prix = prix
    
    #Modificateur, permet de changer la couleur
    def setColor(self, couleur):
        self.couleur = couleur
        
    #Calculateur de TVA (taxe de 21%)
    def tva(self):
        return self.prix * 0.21
    
    #Faire rouler le vehicule
    def rouler(self, distance)
        self.kilometres += distance
\end{python}

On peut ensuite utiliser cette classe pour créer une voiture, et s'en servir.

\begin{python}[caption = Exemple de classe]
voiture = Vehicule(10000)              #Cree une voiture au prix de 10 000 euros
facture = voiture.prix + voiture.tva() #Total a payer
voiture.rouler(200)                    #Roule sur 200km
voiture.setColor('red')                #Peint la voiture en rouge
voiture.rouler(150)                    #Roule sur 150km
print(voiture.distance)                #Affiche 350
\end{python}