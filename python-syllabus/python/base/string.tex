Maintenant que nous connaissons les types natifs, on peut aller un peu plus loin. Il y a quelque chose de très important à retenir : \textbf{En Python, tout est objet!} Mêmes les types de données natifs qu'on a vu précédemment sont des objets. Nous apprendrons plus tard à définir nos propres types de données, c'est-à-dire des \textbf{objets} qui ont été déclarés dans des \textbf{classes}. Dans cette section on va s'attarder sur un type de données qu'on a déjà rencontré : le \texttt{String} (ce qui veut dire \textit{chaîne} en anglais).

Un String est le type qui représente un texte. Tout texte entouré de guillemets est considéré comme un \texttt{String}.

\begin{python}[caption = type \texttt{String}]
#Exemples de Strings
v1 = "Ceci est une phrase"
v2 = ""
v3 = "3" #Est un String et non un int car il y a des guillemets!
v4 = "42.5"
v5 = "qpdjnqpuifnsfvsnvsdv5s1vd1s5v15xv4s5d"
print(type(v1)) #Imprime str (ce qui correspond a String)
\end{python}

\subsection{Opérations sur les \texttt{String}}
Chaque lettre du String est numérotée de par ordre croissant (commençant par zéro). Ainsi le String "Hello" possède la lettre \textit{H} en \texttt{0}, la lettre \textit{e} en \texttt{1}, etc... On accède à une lettre d'un \texttt{String} de la manière suivante :

\begin{python}[caption = Accès à String]
nom_du_string[numero_de_la_lettre] # Accede a la lettre d'un String
#Exemples
my_string = "Bonjour"
lettre_1 = my_string[0] #lettre_1 vaut 'B'
lettre_4 = my_string[3] #lettre_4 vaut 'j'
\end{python}

Attention à ne pas oublier que les numéros des lettres commencent à 0 et non à 1! C'est une faute très courante au début!

Des opérations courantes sur les \texttt{String} sont présentées dans le tableau ci-dessous.

Notez la syntaxe particulière des instructions lower et upper : le nom de la variable, suivi d'un point, suivi du nom de l'opération. Pour l'instant, utilisez-les telles quelles, on en reparlera dans le chapitre sur les classes et objets.

\begin{center}
\begin{tabularx}{\textwidth}{|>{\columncolor[gray]{0.9}} X|X|}
\hline
\rowcolor[gray]{0.8} \bf Opération & Description\\
\hline
\bf len(nom\_string) & Calcule la longueur d'un String \\
\hline
\bf nom\_string.lower() & Retourne le string en minuscule\\
\hline
\bf nom\_string.upper() & Retourne le string en majuscule\\
\hline
\bf str(nom\_variable) & Cast en String\\
\hline
\bf nom\_string1+nom\_string2 & Retourne les 2 String concaténées\\
\hline
\end{tabularx}
\end{center}