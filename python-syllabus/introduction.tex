
Dans le cadre de ce stage, nous allons aborder quelques points de matière majeurs :
\begin{enumerate}
 \item Premièrement, nous allons étudier le langage de programmation \textbf{PYTHON}.% Ce langage va permettre aux étudiants d'interagir avec l'ordinateur et sera utile pour atteindre les objectifs fixés par le stage. 
 Nous reprendrons les bases% - que les élèves sont supposés maîtriser -
 , puis nous approfondirons la matière progressivement. De nombreux exercices accompagneront l'élève durant cet apprentissage.\\
 \item Ensuite, nous découvrirons le \textbf{Raspberry Pi}. Nous commencerons par en expliquer les principes, l'installation et l'utilisation à distance. Enfin, progressivement, les élèves apprendront à programmer le comportement de leur Raspberry Pi.\\
 \item Enfin, nous utiliserons ces concepts pour réaliser un petit jeu vidéo, et des contrôleurs physiques pour celui-ci.\\
\end{enumerate}


%Afin que les élèves ne se sentent pas trop seuls, les exercices seront effectués par groupe de deux. En informatique, nous appelons cette technique le \textit{pair programming}. Une telle méthode de travail permet à la fois de rendre l'apprentissage plus agréable et plus efficace. En effet, cette technique encourage les élèves à dialoguer plus souvent et les aide à sortir de leur isolement écran-utilisateur.  De plus, elle permet aux paires de se corriger mutuellement, tirant ceux qui ont un peu plus de mal vers le haut. Enfin, la programmation par paire est une technique utilisée dans le monde professionnel. C'est donc intéressant d'en donner d'ores et déjà un petit aperçu.

\section{Plan de la semaine}
\begin{tabular}{p{2cm}p{12cm}}
    Lundi &
    \begin{itemize}
        \item Prise de contact
        \item Les bases du Python
        \item Bilan de la journée
    \end{itemize} \\
    Mardi &
    \begin{itemize}
        \item Approfondissement du Python (1)
        \item Exercices récapitulatifs en Python
        \item Bilan de la journée
    \end{itemize} \\
    Mercredi &
    \begin{itemize}
        \item Approfondissement du Python (2)
        \item Introduction au Raspberry Pi et au projet
        \item Bilan de la journée
    \end{itemize} \\
    Jeudi &
    \begin{itemize}
        \item Utilisation du Raspberry Pi
        \item Projet
        \item Bilan de la journée
    \end{itemize} \\
    Vendredi &
    \begin{itemize} 
        \item Projet
        \item Présentation aux parents 
    \end{itemize} \\
\end{tabular}

Ce plan est donné à titre informatif, il pourra être modifié selon les circonstances et les envies du groupe.\\

L'accent sera beaucoup mis sur les exercices. La théorie n'est pas très compliquée, mais c'est en appliquant ce qu'on vient de voir qu'on apprend le mieux. Des exemples sont souvent données à la fin des sections qui suivent; d'autres seront données durant le stage, du plus simple au plus intéressant!